\documentclass[a4paper,12pt]{article} %Tipo de documento


%Insercion de paquetes.
\usepackage[utf8]{inputenc}
\usepackage[spanish]{babel}
\usepackage{graphicx} %Inserción de imágenes.
\usepackage[table,xcdraw]{xcolor} %Detecicon de colores.
\usepackage[most]{tcolorbox} %Para el cuadrado del nombre.
\usepackage{fancyhdr} %Estilo de Página.
\usepackage[hidelinks]{hyperref} %Gestion de hypervinculos.
\usepackage{parskip} %Arreglo de la sangría del documento.
\usepackage{cite}

\usepackage{geometry}
\usepackage{listings}
\usepackage[T1]{fontenc}
\usepackage[backend=biber]{biblatex}


%Variables para insertar codigo.
\definecolor{codegreen}{rgb}{0,0.6,0}
\definecolor{codegray}{rgb}{0.5,0.5,0.5}
\definecolor{codepurple}{HTML}{59b300}
\definecolor{backcolour}{rgb}{0.95,0.95,0.92}
\lstdefinestyle{mystyle}{
    backgroundcolor=\color{backcolour},
    commentstyle=\color{codegreen},
    keywordstyle=\color{magenta},
    numberstyle=\tiny\color{codegray},
    stringstyle=\color{codepurple},
    basicstyle=\ttfamily\footnotesize,
    breakatwhitespace=false,
    breaklines=true,
    captionpos=b,
    keepspaces=true,
    numbers=left,
    numbersep=5pt,
    showspaces=false,
    showstringspaces=false,
    showtabs=false,
    tabsize=2
    }
    \lstset{style=mystyle}
    %Fin variables de codigo
    
    %Variables extras.
    \definecolor{black}{HTML}{000000}
    \definecolor{accentColor}{HTML}{0099BB}
    
    \geometry{
        a4paper,
left=20mm,
right=20mm,
top=20mm,
bottom=20mm,
headheight=17pt,
}
\bibliography{referencias}
\fancyhf{}
\pagestyle{fancy}
\lhead{\Proyecto}
\rhead{\Nombre}
\renewcommand{\lstlistingname}{Conjunto de comandos}
\renewcommand{\labelenumii}{\Roman{enumii}}
\newcommand{\image}[2][1]{\includegraphics[width=#1\textwidth,height=#1\textheight,keepaspectratio]{#2}}
\graphicspath{{img/}} %Setting the graphicspath



%=============================================
\newcommand{\Nombre}{Carlos Sesma Usarralde}
\newcommand{\Proyecto}{Ejercicio Accesibilidad}
%=============================================

%Incio de documento.
\begin{document}

\cfoot{\thepage}
%Inicio de Portada.
\begin{titlepage}
    \begin{center}
        \huge\textbf{\Proyecto}
    \end{center}
    \hfill{}
    %Línea naranja
    {\color{accentColor}\hrule}
    \vfill{}
    \begin{center}
        \huge\textbf{Diseño de interfaces Web}
    \end{center}
    \vfill{}
    \begin{center}
        {\large{} \Nombre}
    \end{center}
\end{titlepage}
%Fin de portada.
%Inicio índice
\clearpage{}
\tableofcontents{}

\thispagestyle{empty}
\clearpage{}
\section{Analiza la accesibilidad de las siguientes web utilizando la herramienta \href{https://wave.webaim.org/}{Wave}}
\subsection{\href{http://www.cpilosenlaces.com}{CPI Los Enlaces}}
En la consulta de la web del CPIFP Los Enlaces he encontrado 4 errores, 8 errores de contraste y 19 alertas.
\subsubsection{Imagenes sin alt }
Error: He encontrado 2 imagenes sin texto alt

Solución: Añadir etiqueta alt para que los lectores de pantalla puedan leerlo en lugar de la imagen

Capturas:

\image{1}
\subsubsection{Links vacios }
Error: He encontrado 2 enlaces que llevan a `\#' (a ningún sitio). Estos links se utilizan para alternar un elemento llamado \textit{advancedmenu}


\subsubsection{Errores de contraste}
La pagina tiene 8 errores de contraste ya que hay 8 elementos que tienen muy poco contraste entre ellos lo cual puede dificultar la lectura para usuarios con problemas de visión.



\subsection{\href{http://educa.aragon.es}{Educa Aragón}}
En la consulta de la web de Educa Aragón he encontrado 5 errores y 8 alertas.


\subsection{\href{https://gdmlamerced.com}{GDM La Merced}}
En la consulta de la web de GDM La Merced he encontrado 6 errores, 7 alertas y 9 errores de contraste.
\subsubsection{Imagenes sin alt}
Error: He encontrado 2 imagenes sin texto alt

Solución: Añadir etiqueta alt para que los lectores de pantalla puedan leerlo en lugar de la imagen

Capturas:

\image{5}

\subsubsection{Links vacios}
Error: La pagina dice que tiene 4 links vacios ya que estos links tienen iconos o elementos h2. Esto la web lo detecta como que no tiene texto.
Solución: Cambiar el orden de los elementos y poner el <a> dentro del h2 o del icono.

Captura:

\image{6}

\section{Analiza la accesibilidad de las mismas paginas del punto anterior mediante otra herramienta de accesibilidad. Explica los errores, posibles soluciones y capturas de pantalla}


\subsection{\href{http://www.cpilosenlaces.com}{CPI Los Enlaces}}


\subsection{\href{http://educa.aragon.es}{Educa Aragón}}
La pagina muestra algun error más que Wave:

\image{9}


\subsection{\href{https://gdmlamerced.com}{GDM La Merced}}
La pagina muestra menos errores y nos muestra lo siguiente:
Captura:

\image{7}

\subsubsection{Los elementos <frame> o <iframe> no tienen título}
Error: Los marcos no tienen title, lo cual no permite a los lectores de pantalla describir el contenido del marco
Solución: Añadir atributo title.

\image{8}


\section{Enumera 3 pruebas que permiten evaluar la accesibilidad de un sitio que puedas realizar, explicar y mostrar con capturas}


\end{document}
